\documentclass[epsfig,10pt,fullpage]{article}

\newcommand{\LabNum}{3}
\newcommand{\CommonDocsPath}{../../../../common/docs}
\addtolength{\textwidth}{1.5in}
\addtolength{\oddsidemargin}{-0.75in}
\addtolength{\topmargin}{-0.75in}
\addtolength{\textheight}{1.5in}
\addtolength{\evensidemargin}{0.75in}
\setlength\parindent{0pt}
\raggedbottom

\usepackage{ae,aecompl}
\usepackage{epsfig,float,times}
\usepackage[hypcap]{caption}
\usepackage[pdftex, colorlinks]{hyperref}
\usepackage{graphicx}
\usepackage[usenames, dvipsnames]{color}
\usepackage{rotating}
\usepackage{tikz}
\usetikzlibrary{automata,positioning}
\usepackage{placeins}

\widowpenalty 10000
\clubpenalty 10000

\newcommand{\red}[1]{{\color{red}\sf{#1}}}
\newcommand{\green}[1]{{\color{green}\sf{#1}}}
\newcommand{\blue}[1]{{\color{blue}\sf{#1}}}
\definecolor{PineGreen}{rgb}{0.0, 0.47, 0.44}
\definecolor{ForestGreen}{rgb}{0.13, 0.55, 0.13}
\definecolor{Brown}{rgb}{0.59, 0.29, 0.0}

\newcommand{\UPDatePublished}{Oct 2021}
\newcommand{\versnum}{21.1} %version number quartus/AMP
\newcommand{\quartusname}{Quartus\textsuperscript{\textregistered} Prime}	
\newcommand{\UPTextBar}{For \quartusname{} \versnum{}}
\newcommand{\thisyear}{2021 } %for copyright
\newcommand{\company}{FPGAcademy.org}
\newcommand{\longteamname}{FPGAcademy.org}
\newcommand{\teamname}{FPGAcademy}
\newcommand{\website}{FPGAcademy.org}

\newcommand{\productAcronym}{AMP}
\newcommand{\productNameShort}{Monitor Program}

\newcommand{\productNameMedTM}{A Monitor Program}
\newcommand{\productNameMed}{A Monitor Program}

%\newcommand{\headerLogoFilePath}[1]{#1/FPGAcademy.png}

% listings is a package that supports encapsulating source code in LaTeX conveniently
\usepackage{listings}

\def\expandparam\lstinputlisting[#1]#2{\edef\tmp{\noexpand\lstinputlisting[#1]{#2}}\tmp}

%%%%%%%%%%%%%%%%%%%% Source Code Formatting %%%%%%%%%%%%%%%%%%%%
\definecolor{globalCommentColour}{rgb}{0.588,0.588,0.588}

%%%%%%%%%%%%%%%%%%%%%%%%%%%%%%%%%%%%%%%%%%%%%%%%%%%%
% Defining language style
% NiosII ASM
\lstdefinelanguage[NiosII]{Assembler} {
  morekeywords={add, addi, and, andhi, andi, beq, bge, bgeu, bgt, bgtu, ble,  bleu, blt, bltu, bne, br, break,
  bret, call, callr, cmpeq, cmpeqi, cmpge, cmpgei, cmpgeu, cmpgeui, cmpgt, cmpgti, cmpgtu, cmpgtui, cmple,
  cmplei, cmpleu, cmpleui, cmplt, cmplti, cmpltu, cmpltui, cmpne, cmpnei, custom, div, divu, eret, flushd,
  flushda, flushi, flushp, initd, initda, initi, jmp, jmpi, ldb, ldbio, ldbu, ldbuio, ldh, ldhio, ldhu, ldhuio,
  ldw, ldwio, mov, movhi, movi, movia, movui, mul, muli, mulxss, mulxsu, mulxuu, nextpc, nop, nor, or, orhi, ori,
  rdctl, rdprs, ret, rol, roli, ror, sll, slli, sra, srai, srl, srli, stb, stbio, sth, sthio, stw, stwio,
  sub, subi, sync, trap, wrctl, wrtcl, wrprs, xor, xori, xorhi, xori},
  morekeywords=[2]{.abort, .ABORT, .align, .app-file, .ascii, .asciz, .balign, .byte, .comm, .data, .def,
  .desc, .dim, .double, .eject, .else, .end, .endef, .endif, .equ, .equiv, .err, .extern, .file, .fill, .float,
  .global, .globl, .hword, .ident, .if, .include, .int, .irp, .irpc, .lcomm, .lflags, .line, .linkonce, .ln,
  .list, .long, .macro, .mri, .nolist, .octa, .org, .p2align, .psize, .quad, .rept, .sbttl, .scl, .section,
  .set, .short, .single, .size, .sleb128, .skip, .space, .stadb, .stabn, .stabs, .string, .symver, .tag,
  .text, .title, .type, .val, .uleb128, .word},
  morekeywords=[3]{et, bt, gp, sp, fp, ea, sstatus, ra, pc, status, estatus, bstatus, ienable, ipending, cpuid,
  exception, pteaddr, tlbacc, tlbmisc, eccinj, badaddr, config, mpubase, mpuacc},
  sensitive=t,
  alsoletter=.,
  morestring=[b]",
  morecomment=[s]{/*}{*/},
  morecomment=[l]\#,
}[keywords,comments,strings]
   
%% NOTE: morekeywords=[2] are GNU directives.
   
\definecolor{niosInstructionColour}{rgb}{0.000,0.608,0.000}
\definecolor{niosDirectiveColour}{rgb}{0.000,0.000,0.902}
\definecolor{niosSpecialRegColour}{rgb}{0.000,0.000,0.000}
\definecolor{niosStringColour}{rgb}{0.808,0.482,0.000}
   
%% NOTE: To make bold use: =\bfseries\color{<colour>}
\lstdefinestyle{defaultNiosStyle} {
  language=[NiosII]{Assembler},
  stringstyle=\color{niosStringColour},
  keywordstyle=\color{niosInstructionColour},
  keywordstyle=[2]\color{niosDirectiveColour},
  keywordstyle=[3]\itshape\color{niosSpecialRegColour}
}
%%%%%%%%%%%%%%%%%%%%%%%%%%%%%%%%%%%%%%%%%%%%%%%%%%%%

%%%%%%%%%%%%%%%%%%%%%%%%%%%%%%%%%%%%%%%%%%%%%%%%%%%%
% Defining language style
% ArmA9 ASM
\lstdefinelanguage[ArmA9]{Assembler} {
  morekeywords={ADC, ADD, ADDS, AND, ANDS, B, BAL, BEQ, BGE, BGT, BL, BLT, BIC, BKPT, BLX, BNE, BX, CDP, CLZ, CMN, CMP, EOR,
  EORS, LDC, LDM, LDR, LDRB, LDRBT, LDRH, LDRSB, LDRSH, LDRT, LSL, MCR, MLA, MOV, MOVW, MOVT, MRC, MRS, MSR, MUL, MVN, ORR, PLD,
  ROR, RSB, RSC, SBC, SMLAL, SMULL, STC, STM, STR, STRB, STRBT, STRH, STRT, SUB, SUBS, SWI, SWP, SWPB, TEQ, UMLAL,
  PUSH, POP, MOVS, RORS, LSR},
  morekeywords=[2]{.abort, .ABORT, .align, .app-file, .ascii, .asciz, .balign, .byte, .comm, .data, .def,
  .desc, .dim, .double, .eject, .else, .end, .endef, .endif, .equ, .equiv, .err, .extern, .file, .fill, .float,
  .global, .globl, .hword, .ident, .if, .include, .int, .irp, .irpc, .lcomm, .lflags, .line, .linkonce, .ln,
  .list, .long, .macro, .mri, .nolist, .octa, .org, .p2align, .psize, .quad, .rept, .sbttl, .scl, .section,
  .set, .short, .single, .size, .sleb128, .skip, .space, .stadb, .stabn, .stabs, .string, .symver, .tag,
  .text, .title, .type, .val, .vectors, .uleb128, .word},
  morekeywords=[3]{SP, PC, MIDR, CTR, TCMTR, TLBTR, MPIDR, ID_PFR0, ID_PFR1, ID_DFR0, ID_MMFR0, ID_MMFR1, ID_MMFR2,
  ID_MMFR3, ID_ISAR0, ID_ISAR1, ID_ISAR2, ID_ISAR3, ID_ISAR4, CCSIDR, CLIDR, AIDR, CSSELR, TTBR0, TTRB1, TTBR2, DACR,
  DFSR, IFSR, ADFSR, AIFSR, DFAAR, IFAR, ICIALLUIS, BPIALLIS, PAR, ICIALLU, ICIMVAU, BPIALL, DCIMVAC, DCISW, V2PCWPR,
  DCCVAC, DCCSW, DDIMVAC, DCISW, TLBALLIS, TLBIMVAIS, TLBIASIDIS, TLBIMVAAIS, TLBIALL, TLBIMVA, TLBIASID, TLBIMVAA,
  PMCR, PMCNTENSET, PMCNTENCLR, PMOVSR, PMSWINC, PMSELR, PMXEVTYPER, PMXEVCNTR, PMUSERENR, PMINTENSET, PMINTENCLR,
  PRRR, NRRR, PLEIDR, PLEASR, PLEFSR, PLEUAR, PLEPCR, VBAR, MVBAR, ISR, FCSEIDR, CONTEXTIDR, TPIDRURW, TPIDRURO, TPIDRPRW},
  sensitive=f,
  alsoletter=.,
  morestring=[b]",
  morecomment=[s]{/*}{*/},
  morecomment=[l]{//},
}[keywords,comments,strings]
   
%% NOTE: morekeywords=[2] are GNU directives.
   
\definecolor{armInstructionColour}{rgb}{0.000,0.608,0.000}
\definecolor{armDirectiveColour}{rgb}{0.000,0.000,0.902}
\definecolor{armSpecialRegColour}{rgb}{0.000,0.000,0.000}
\definecolor{armStringColour}{rgb}{0.808,0.482,0.000}
   
\lstdefinestyle{defaultArmStyle} {
  language=[ArmA9]{Assembler},
  stringstyle=\color{armStringColour},
  keywordstyle=\color{armInstructionColour},
  keywordstyle=[2]\color{armDirectiveColour},
  keywordstyle=[3]\itshape\color{armSpecialRegColour}
}
%%%%%%%%%%%%%%%%%%%%%%%%%%%%%%%%%%%%%%%%%%%%%%%%%%%%

%%%%%%%%%%%%%%%%%%%%%%%%%%%%%%%%%%%%%%%%%%%%%%%%%%%%
% Defining language style
% FPGAcademy ASM
\lstdefinelanguage{ASM}{
  morekeywords = [1]{mv, mvt, mvne, mvcc, add, sub, st, ld, and, b, bne, beq, bcc, bcs},
  morekeywords = [2]{word, define},
  keywordstyle = [1]\color{ForestGreen},
  keywordstyle = [2]\color{blue},
  sensitive = true,
  morecomment = [l]{//},
}

\lstset{
  language = ASM,
  basicstyle=\small\color{black}\ttfamily,
  commentstyle=\small\color{Brown}\itshape\ttfamily,
  showstringspaces=false,
  frame=none, %lines % boxed listings
  breaklines=true,
  breakatwhitespace=true,
  tabsize=3
}
%%%%%%%%%%%%%%%%%%%%%%%%%%%%%%%%%%%%%%%%%%%%%%%%%%%%

%%%%%%%%%%%%%%%%%%%%%%%%%%%%%%%%%%%%%%%%%%%%%%%%%%%%
% Defining language style
% Java
\definecolor{javaStringColour}{rgb}{0.808,0.482,0}
%%%%%%%%%%%%%%%%%%%%%%%%%%%%%%%%%%%%%%%%%%%%%%%%%%%%

%%%%%%%%%%%%%%%%%%%%%%%%%%%%%%%%%%%%%%%%%%%%%%%%%%%%
% Defining language style
% C
\definecolor{CStringColour}{rgb}{0.808,0.482,0}

\lstset{
  language = C,
  basicstyle=\small\color{black}\ttfamily, 
  commentstyle=\small\color{PineGreen}\itshape\ttfamily,
  keywordstyle=\small\color{blue}\bfseries\ttfamily,
  showstringspaces=false,
  frame=none, %lines % boxed listings
  breaklines=true,
  breakatwhitespace=true,
  tabsize=3
}
%%%%%%%%%%%%%%%%%%%%%%%%%%%%%%%%%%%%%%%%%%%%%%%%%%%%

%%%%%%%%%%%%%%%%%%%%%%%%%%%%%%%%%%%%%%%%%%%%%%%%%%%%
% Defining language style
% Verilog
\definecolor{verilogCommentColour}{rgb}{0.000,0.502,0.000}

\lstdefinestyle{defaultVerilogStyle} {
  language={Verilog},
  keywordstyle=\color{blue},
  commentstyle=\color{verilogCommentColour}
}
%%%%%%%%%%%%%%%%%%%%%%%%%%%%%%%%%%%%%%%%%%%%%%%%%%%%

%%%%%%%%%%%%%%%%%%%%%%%%%%%%%%%%%%%%%%%%%%%%%%%%%%%%
% Defining language style
% VHDL
\lstdefinestyle{defaultVHDLStyle} {
  language={VHDL},
  keywordstyle=\color{blue},
  commentstyle=\color{verilogCommentColour}
}
%%%%%%%%%%%%%%%%%%%%%%%%%%%%%%%%%%%%%%%%%%%%%%%%%%%%

%%%%%%%%%%%%%%%%%%%%%%%%%%%%%%%%%%%%%%%%%%%%%%%%%%%%
% Defining language style
% LaTeX
\lstdefinelanguage[LocalLaTeX]{TeX}[LaTeX]{TeX}{moretexcs={bf, it, sf, lstset},}

\lstdefinestyle{defaultLocalLatexStyle} {
  language=[LocalLatex]{TeX},
  keywordstyle=\color{blue}\bfseries,
  keywordstyle=[2]\color{blue},
  keywordstyle=[3]\color{blue}\bfseries
}
%%%%%%%%%%%%%%%%%%%%%%%%%%%%%%%%%%%%%%%%%%%%%%%%%%%%

%%%%%%%%%%%%%%%%%%%%%%%%%%%%%%%%%%%%%%%%%%%%%%%%%%%%
% Defining language style
% Default
\lstset{
  basicstyle=\small\color{black}\ttfamily,
  commentstyle=\small\color{globalCommentColour}\itshape\ttfamily,
  keywordstyle=\small\color{blue}\bfseries\ttfamily,
  showstringspaces=false,
  frame=none, %lines % boxed listings
  breaklines=true,
  breakatwhitespace=true,
  tabsize=3
}
%%%%%%%%%%%%%%%%%%%%%%%%%%%%%%%%%%%%%%%%%%%%%%%%%%%%


\hypersetup{
  pdftitle={Computer Organization Lab Exercise \LabNum},
  linkcolor=blue,
  hyperindex=true,
  pdfauthor={FPGAcademy.org},
  pdfkeywords={FPGAcademy.org, FPGAcademy, Lab, Exercise, Computer Organization},
  bookmarks,
  bookmarksopen=false,
  filecolor=blue,
  pdfstartview={FitH},
  urlcolor=blue,
  plainpages=false,
  pdfpagelabels=true,
  linkbordercolor={1 1 1} %no color for link border
}



\begin{document}

\centerline{\huge Computer Organization}
~\\
\centerline{\huge Laboratory Exercise \LabNum}
~\\
\centerline{\large Subroutines and Stacks}
~\\

This exercise is about subroutines and subroutine linkage using the Nios\textsuperscript{\textregistered} II processor.  You will 
learn about the concepts of parameter passing, stacks, and recursion.  For this 
exercise you have to know the Nios~II processor architecture and its assembly language, and 
you should have a basic understanding of the C programming language. 

\section*{Part I}
\addcontentsline{toc}{1}{Part I}
You are to write a Nios II assembly language subroutine called FINDSUM that uses a loop to 
compute the summation
$\displaystyle\sum_{i=1}^{N} i$.
~~Equivalent code for your subroutine in the C language is shown below. 

~\\
\begin{minipage}[t]{16.5 cm}
\lstinputlisting[language=C]{../design_files/find_sum/find_sum.c}
\end{minipage}

~\\
\noindent
You need to provide a main program that calls your FINDSUM subroutine.
Part of this main program is shown below.
The value of argument $N$ that is used for your subroutine is stored
in memory as shown in the code---your main program needs to load this value from memory and pass 
it to the subroutine, using processor register R0. Return the result from the subroutine
in the same register, R0.

~\\
\begin{minipage}[t]{16.5 cm}
\lstinputlisting[style=defaultNiosStyle]{../design_files/find_sum/find_sum.s}
\end{minipage}

\newpage
\noindent
Perform the following:

\begin{enumerate}
\item
Create a new folder and make a Monitor Program project for your summation code. Depending on your DE-series board, use the corresponding computer system 
listed in Table~\ref{tab:computer_systems}.

\begin{table}[H]
	\begin{center}
	\begin{tabular}{ l | l }
	\bf{Board} & \bf{Computer System} \\
	\hline
	\rule{0pt}{3ex}DE0-Nano & DE0-Nano Computer \\
	DE0-Nano-SoC & DE0-Nano-SoC Computer \\
	DE10-Nano & DE10-Nano Computer \\
	\end{tabular}
	\caption{DE-series board computer systems}
	\label{tab:computer_systems}
	\end{center}
\end{table}

\item
Assemble and download your program.  Test it for various values of $N$.
\end{enumerate}

~\\
\noindent
{\bf Part II}
~\\
~\\
\noindent
You are to write an assembly language program that sorts a list of 32-bit numbers 
into descending order.  The first entry in the list gives the number of data elements to 
be sorted, and the rest of the list provides the data. The list of data must be 
sorted ``in place'', meaning that you are not allowed to create a copy in memory of the 
list to do the sorting.  

~\\
\noindent
A program written in the C language that performs the required sorting operation is shown in
Figure~\ref{fig:C_code}. This program implements a simple bubble-sort algorithm. It uses
an outer loop to traverse the list a number of times until sorted. An inner loop 
calls the {\it SWAP} subroutine, not shown in the figure, which swaps the list elements in
memory when needed.

~\\
\noindent
Write a main program and subroutine in the Nios II assembly language that is equivalent to the C code 
in Figure~\ref{fig:C_code}.
Your SWAP subroutine should be passed the address of a list element in processor register R0, 
and should provide its return value to the main program in the same register. 

\begin{figure}[H]
\begin{center}
\begin{minipage}[t]{16.5 cm}
\lstinputlisting[language=C, linerange=5-23]{../design_files/bubble_C/bubble.c}
\end{minipage}
\end{center}
\caption{A bubble-sort algorithm.}
\label{fig:C_code}
\end{figure}

~\\
\noindent
The list can be defined as part of the data for your assembly language program as follows:

\begin{lstlisting}[style=defaultNiosStyle]
List: 	.word 	10, 0, 1, 2, 3, 4, 5, 6, 7, 8, 9
\end{lstlisting}

\noindent
Perform the following:

\begin{enumerate}
\item Create a new folder and make a Monitor Program project for your sorting code. Select
the Nios II processor and use the appropriate Computer for your DE-series board.

\item
Test your algorithm with various data sets and ensure
that the list of data is properly sorted in-place in the memory. A good debugging
technique for this code is to use the Memory tab in the Monitor Program to view the
contents of the list as the sorting algorithm progresses. Each time a breakpoint is reached by
the processor (or an instruction is single-stepped), the list can be examined to see how
the items are being swapped.
\end{enumerate}

~\\
~\\
\noindent
{\bf Part III}
~\\
~\\
For this part of the exercise you are to rewrite your FINDSUM subroutine from Part I to make 
it recursive.  Equivalent C code for your subroutine is shown below.

~\\
\begin{minipage}[h]{16.5 cm}
\lstinputlisting[language=C, linerange=14-20]{../design_files/recurse_sum/recurse.c}
\end{minipage}

~\\
~\\
\noindent
In your assembly-language code, make sure to initialize the Nios II stack pointer to 
a suitable value, and use the stack to save the state of the FINDSUM subroutine each 
time that it recurses.

~\\
\noindent
Create a new folder and make a Monitor Program project for your recursive code.
Assemble, download, and test your program.

~\\
~\\
\noindent
{\bf Part IV}
~\\
~\\
\noindent
You are to write an assembly-language subroutine that computes the $n^{th}$ number in 
the Fibonacci sequence.  

\noindent The $n^{th}$ Fibonacci number is computed as 
$$
Fib(n) = Fib(n - 1) + Fib(n - 2)
$$
\noindent
Note that {\it Fib}(0) = 0 and {\it Fib}(1) = 1.

~\\
\noindent
Your subroutine has to be recursive. Equivalent C code for such a subroutine is shown below. 

~\\
\begin{minipage}[H]{16.5 cm}
\lstinputlisting[language=C, linerange=16-22]{../design_files/recurse_Fib/recurse.c}
\end{minipage}

~\\
\noindent
You need to provide a main program that calls your FIBONNACI subroutine, in the same way as for
the earlier parts of the exercise. The value of the argument $N$ should be loaded from 
memory and passed to your subroutine.  You can assume that $N > 1$.

~\\
\noindent
Make sure to initialize the Nios II stack pointer to 
a suitable value, and use the stack to save the state of the FIBONNACI subroutine each 
time that it recurses.

~\\
\noindent
Create a new folder and make a Monitor Program project for your Fibonacci code.
Assemble, download, and test your program.


%%%%%%%%%%%%%%%%%%%%%%%%%%%%%%%%%%%%%%%%
%%% FPGAcademy Copyright Information %%%
%%%%%%%%%%%%%%%%%%%%%%%%%%%%%%%%%%%%%%%%

%Always put the copyright on a new page (clear page), with some vertical space from top
\clearpage
\vspace{1in}

\noindent

Copyright {\copyright} FPGAcademy.org. All rights reserved. FPGAcademy and the 
FPGAcademy logo are trademarks of FPGAcademy.org.  This document is provided 
"as is", without warranty of any kind, express or implied, including but not 
limited to the warranties of merchantability, fitness for a particular purpose 
and noninfringement. In no event shall the authors or copyright holders be 
liable for any claim, damages or other liability, whether in an action of 
contract, tort or otherwise, arising from, out of or in connection with the 
document or the use or other dealings in the document.
~\\
~\\
**Other names and brands may be claimed as the property of others.


\end{document}
